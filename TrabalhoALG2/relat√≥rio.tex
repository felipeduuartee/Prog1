\documentclass[conference]{IEEEtran}
\usepackage[portuguese]{babel}
\usepackage{cite}
\usepackage{amsmath,amsfonts}
\usepackage{graphicx}
\usepackage{textcomp}
\usepackage{pgfplots}
\usetikzlibrary{plotmarks}
%\usepackage{nolist}
\usepackage{changepage}
\usepackage{float}
\usepackage{subfig}
\usepackage{amsmath}
\usepackage{tabularx}


\pgfplotsset{compat=1.16}
\pgfkeys{/pgf/number format/.cd,1000 sep={}}
\pgfplotsset{standard/.style={
    title style={font=\bfseries, yshift=-0.5ex},
    xlabel style={font=\bfseries},
    ylabel style={font=\bfseries},
    xmode=log,
    ymode=log,
    legend pos=north west,
    grid=major,
    width=\linewidth,
    height=3.5cm
}}

\begin{document}

\title{Estudo Sistemático de Algoritmos de Busca e Ordenação em Diferentes Contextos}

\author{\IEEEauthorblockN{Felipe Duarte Silva\\ 
\textit{Universidade Federal do Paraná -- UFPR}\\
felipeduarte@ufpr.br}

\maketitle

\begin{abstract}
Este relatório aborda e compara diferentes algoritmos de busca e ordenação, considerando sua eficiência, número de comparações e tempo de execução para diferentes tamanhos de entradas.
\end{abstract}

\begin{IEEEkeywords}
Algoritmos, Busca Sequencial, Busca Binária, Insertion Sort, Selection Sort, Merge Sort, Quick Sort, Análise de eficiência e Comparação de desempenho.
\end{IEEEkeywords}

\section{Introdução}
Este relatório investiga versões recursivas de cinco algoritmos de ordenação e dois de busca implementados em C. Para simular os piores casos, utilizamos vetores ordenados de maneira não crescente. Enquanto analisamos a performance da maioria dos algoritmos de ordenação até um bilhão de posições, o Insertion Sort e o Selection Sort foram limitados a cem mil. Os resultados são apresentados através de gráficos que relacionam o número de comparações, tempo de execução e tamanho do vetor.

\section{Algoritmos de Busca}
\subsection{Busca Sequencial}
A Busca Sequencial, consiste em uma técnica que analisa cada elemento da lista de dados sequencialmente. O processo continua até que o elemento desejado seja encontrado ou até que todos os elementos tenham sido verificados, indicando a ausência do elemento na lista.

\subsection{Busca Binária}
A Busca Binária é uma técnica de busca que tem como pré-requisito que a lista de dados esteja previamente ordenada. O algoritmo divide o conjunto de dados pela metade repetidamente, eliminando a metade onde o item não pode estar presente, até que o elemento desejado seja localizado ou o conjunto esteja vazio.

\section{Algoritmos de Ordenação}
\subsection{Insertion Sort}
O algoritmo Insertion Sort, percorre sequencialmente a lista de dados, posicionando cada elemento na posição adequada em relação aos elementos já avaliados, garantindo a ordenação até o elemento atual.

\subsection{Selection Sort}
O algoritmo Selection Sort identifica o menor (ou maior) elemento da lista e troca-o com o primeiro elemento não ordenado. O processo é repetido até que todos os elementos estejam ordenados.

\subsection{Merge Sort}
O Merge Sort particiona a lista de dados em duas metades, ordena-as individualmente e, posteriormente, mescla as duas metades ordenadas para produzir a lista final ordenada.

\subsection{Quick Sort}
O Quick Sort seleciona um elemento chamado 'pivô' e particiona a lista de modo que todos os elementos menores que o pivô venham antes dele, e todos os elementos maiores venham depois. Este processo é aplicado às sublistas resultantes.

\section{Resultados dos Testes}
A avaliação do desempenho dos algoritmos foi realizada usando vetores configurados em ordem não crescente e testados em uma máquina com processador 11th Gen Intel® Core™ i7-1165G7 com velocidade base de 2.80GHz e um total de 15GiB de RAM. Sistema Operacional: Ubuntu 20.04.6. Os resultados são apresentados abaixo graficamente ilustrados com números de comparações aproximados.


\vspace{0.4cm}



\textbf{Insertion Sort}: Custo de \(C(n) = ((n(n - 1))/2)\) comparações e um tempo de execução que varia linearmente com o tamanho, conforme ilustrado na figura 1.
\begin{figure}[H]
 %\begin{adjustwidth}{-0.1cm}{}
    \centering
        \begin{tikzpicture}
            \begin{axis}[
                title={Comparações vs. Tamanho},
                xlabel={Tamanho},
                ylabel={Comparações},
                xtick={1000,50000,100000},
                xticklabels={$10^3$,$5 \times 10^4$,$10^5$},
                ytick={5*10^5,1.25*10^9,5*10^9},
                yticklabels={$5 \times 10^5$,$1.25 \times 10^9$,$5 \times 10^9$},
                xmode=log,
                ymode=log,
                legend pos=north west,
                grid=major,
                width=\linewidth,
                height=3.8cm,
                width=8cm
            ]

            \addplot[blue,mark=*,dashed] coordinates {
                (1000, 5*10^5)
                (50000, 1.25*10^9)
                (100000, 5*10^9)
            };
            \end{axis}
        \end{tikzpicture}
        %\end{adjustwidth}

    \hfill
        \begin{tikzpicture}
         \begin{adjustwidth}{-0.2cm}{}

            \begin{axis}[
                title={Tempo vs. Tamanho},
                xlabel={Tamanho},
                ylabel={Tempo (s)},
                xtick={1000,50000,100000},
                xticklabels={$10^3$,$5 \times 10^4$,$10^5$},
                ytick={0.004044,3.835591,14.897176},
                yticklabels={0.004,3.836,14.897},
                xmode=log,
                ymode=log,
                legend pos=north west,
                grid=major,
                width=\linewidth,
                height=3.8cm,
                width=8cm
            ]
            \addplot[red,mark=*,dashed] coordinates {
                (1000, 0.004044)
                (50000, 3.835591)
                (100000, 14.897176)
            };
            \end{axis}
       \end{adjustwidth}
        \end{tikzpicture}

    \caption{Desempenho do Insertion Sort: comparações e tempo de execução quanto ao tamanho do vetor}
    \label{fig:insertionSortPerformanceGraph}
\end{figure}















\textbf{Selection Sort}: Custo de \(C(n) = ((n(n - 1))/2)\) comparações e um tempo de execução que varia linearmente com o tamanho, conforme ilustrado na figura 2.

\begin{figure}[H]
 \begin{adjustwidth}{-1.9cm}{}
    \centering
        \begin{tikzpicture}
            \begin{axis}[
                title={Comparações vs. Tamanho},
                xlabel={Tamanho},
                ylabel={Comparações},
                xtick={1000,50000,100000},
                xticklabels={$10^3$,$5 \times 10^4$,$10^5$},
                ytick={5*10^5,1.25*10^9,5*10^9},
                yticklabels={$5 \times 10^5$,$1.25 \times 10^9$,$5 \times 10^9$},
                xmode=log,
                ymode=log,
                legend pos=north west,
                grid=major,
                width=\linewidth,
                height=4cm,
                width=8cm
            ]

            \addplot[black,mark=*,dashed] coordinates {
                (1000, 5*10^5)
                (50000, 1.25*10^9)
                (100000, 5*10^9)
            };
            \end{axis}
        \end{tikzpicture}
        \end{adjustwidth}

        \vspace{0.4CM}
    \hfill
        
        \begin{adjustwidth}{-0.3cm}{}
        \begin{tikzpicture}
        
            \begin{axis}[
                title={Tempo vs. Tamanho},
                xlabel={Tamanho},
                ylabel={Tempo (s)},
                xtick={1000,50000,100000},
                xticklabels={$10^3$,$5 \times 10^4$,$10^5$},
                ytick={0.003,3.227,13.042},
                yticklabels={0.003,3.227,13.042},
                xmode=log,
                ymode=log,
                legend pos=north west,
                grid=major,
                width=\linewidth,
                height=4cm,
                width=8cm
            ]
            \addplot[brown,mark=*,dashed] coordinates {
                (1000, 0.003)
                (50000, 3.227)
                (100000, 13.042)
            };
            \end{axis}
        \end{tikzpicture}
    \end{adjustwidth}
    \caption{Desempenho do Selection Sort: comparações e tempo de execução quanto ao tamanho do vetor}
    \label{fig:insertionSortPerformanceGraph}
\end{figure}






























\textbf{Merge Sort}: Custo de aproximadamente \(C(n) = (n\times log n)\) comparações e um tempo de execução que varia linearmente com o tamanho, conforme ilustrado na figura 3.

\begin{figure}[!htpb]
 \begin{adjustwidth}{-2cm}{}
    \centering
    \begin{tikzpicture}
        \begin{axis}[
            title={Comparações vs. Tamanho},
            xlabel={Tamanho},
            ylabel={Comparações},
            xtick={1000000,100000000,1000000000},
            xticklabels={$10^6$,$10^8$,$10^9$},
            ytick={9884992,1314447104,14846928128},
            yticklabels={$9.88 \times 10^6$,$1.31 \times 10^9$,$14.85 \times 10^9$},
            xmode=log,
            ymode=log,
            legend pos=north west,
            grid=major,
            width=\linewidth,
            height=3.5cm,
            width=8cm
        ]
        \definecolor{darkgreen}{rgb}{0.0, 0.5, 0.0}
        \addplot[darkgreen,mark=*,dashed] coordinates {
            (1000000, 9884992)
            (100000000, 1314447104)
            (1000000000, 14846928128)
        };
        \end{axis}
    \end{tikzpicture}
        \end{adjustwidth}
    \hfill
            \vspace{0.2CM}

         \begin{adjustwidth}{-0.48cm}{}
    \begin{tikzpicture}
        \begin{axis}[
            title={Tempo vs. Tamanho},
            xlabel={Tamanho},
            ylabel={Tempo (s)},
            xtick={1000000,100000000,1000000000},
            xticklabels={$10^6$,$10^8$,$10^9$},
            ytick={0.173014,18.439194,205.011166},
            yticklabels={0.173,18.439,205.011},
            xmode=log,
            ymode=log,
            legend pos=north west,
            grid=major,
            width=\linewidth,
            height=3.58cm,
            width=8cm
        ]
        \addplot[gray,mark=*,dashed] coordinates {
            (1000000, 0.173014)
            (100000000, 18.439194)
            (1000000000, 205.011166)
        };
        \end{axis}
    \end{tikzpicture}
    \end{adjustwidth}
    \caption{Desempenho do Merge Sort: comparações e tempo de execução quanto ao tamanho do vetor}
    \label{fig:mergeSortPerformanceGraph}
\end{figure}

\pagebreak


















\textbf{Quick Sort}: Para a escolha do pivô no valor do meio de um vetor preenchido em ordem não crescente, o custo é de aproximadamente \(C(n) = (n\times log n)\) comparações e um tempo de execução de que varia linearmente com o tamanho, conforme ilustrado na figura 4.
\begin{figure}[H]
 \begin{adjustwidth}{0cm}{}
    \centering
        \begin{tikzpicture}
            \begin{axis}[
                title={Comparações vs. Tamanho},
                xlabel={Tamanho},
                ylabel={Comparações},
                xtick={1000000,100000000,1000000000},
                xticklabels={$10^6$,$10^8$,$10^9$},
                ytick={18401192,2511626270,28367876889},
                yticklabels={$1.84 \times 10^7$,$2.51 \times 10^9$,$28.37 \times 10^9$},
                xmode=log,
                ymode=log,
                legend pos=north west,
                grid=major,
                width=\linewidth,
                height=3.5cm,
                width=8cm
            ]
            \definecolor{darkpurple}{rgb}{0.5, 0.0, 0.5}
            \addplot[darkpurple,mark=*,dashed] coordinates {
                (1000000, 18401192)
                (100000000, 2511626270)
                (1000000000, 28367876889)
            };
            \end{axis}
        \end{tikzpicture}
            \end{adjustwidth}
        %\caption{Desempenho do Quick Sort: Comparações em relação ao tamanho do vetor}
    \hfill
            

         \begin{adjustwidth}{0.58cm}{}
        \begin{tikzpicture}
            \begin{axis}[
                title={Tempo vs. Tamanho},
                xlabel={Tamanho},
                ylabel={Tempo (s)},
                xtick={1000000,100000000,1000000000},
                xticklabels={$10^6$,$10^8$,$10^9$},
                ytick={0.085517,11.277285,123.930472},
                yticklabels={0.086,11.277,123.930},
                xmode=log,
                ymode=log,
                legend pos=north west,
                grid=major,
                width=\linewidth,
                height=3.5cm,
                width=8cm
            ]
            \definecolor{darkyellow}{rgb}{0.9, 0.83, 0.0}
            \addplot[darkyellow,mark=*,dashed] coordinates {
                (1000000, 0.085517)
                (100000000, 11.277285)
                (1000000000, 123.930472)
            };
            \end{axis}
        \end{tikzpicture}
    \end{adjustwidth}
    \caption{Desempenho do Quick Sort: comparações e tempo de execução quanto ao tamanho do vetor. }
    \label{fig:quickSortPerformanceGraph}
\end{figure}
\vspace{-0.4cm}
{\small \textit{Nota: Gráficos em escala log-base-10. Assim, comportamentos \(n^2\) ou \(n \times \log n\) são linhas retas.}}

\section{Algoritmos de Busca}

\textbf{Algoritmos de Busca}: As tabelas a seguir representam o número de comparações e o tempo de execução que cada algoritmo demorou para chegar na conclusão de que o elemento buscado não está nos vetores de diferentes tamanhos.

\subsection*{Busca Sequencial}
%\vspace{0.3cm}
\begin{tabularx}{0.5\textwidth}{X|X|X}
\hline
\textbf{Tamanho} & \textbf{Tempo (s)} & \textbf{Comparações} \\

$10^5$ & 0.0011 & $10^5$ \\

$10^4$ & 0.0001 & $10^4$ \\

$10^3$ & 0.0000 & $10^3$ \\
\hline
\end{tabularx}
\hspace{1cm} % Espaço entre as tabelas

\subsection*{Busca Binária}
%\vspace{0.3cm}
\begin{tabularx}{0.5\textwidth}{X|X|X}
\hline
\textbf{Tamanho} & \textbf{Tempo (s)} & \textbf{Comparações} \\

$1.41\times10^9$ & 0.000003 & 31 \\

$10^9$ & 0.000002 & 30 \\

$10^8$ & 0.000001 & 27 \\
\hline
\end{tabularx}

\vspace{0.5cm}

\section{Conclusão}
Os resultados destacam a variabilidade de eficiência entre os algoritmos em diferentes contextos. Comprovou-se que o Quick Sort, influenciado pela escolha do pivô, e o Merge Sort são mais eficientes que o Insertion Sort e o Selection Sort, independentemente da dimensão do vetor. Quanto às técnicas de busca, a Busca Binária, que necessita de dados ordenados, supera a Busca Sequencial em eficiência para vetores de todos os tamanhos.

\end{document}
